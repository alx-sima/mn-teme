% Copyright (C) 2023 Alexandru Sima (312CA) %
\documentclass{article}

\usepackage[hidelinks]{hyperref}
\usepackage[romanian]{babel}
\usepackage{fancyhdr}
\usepackage{graphicx}


\begin{document}

\pagestyle{fancy}
\begin{titlepage}
      \begin{center}
            Universitatea POLITEHNICA din București \\
            Facultatea de Automatică și Calculatoare \\[5px]
            \includegraphics[width=1.2cm]{upb.jpg}
            \includegraphics[width=1.2cm]{acs.jpg}

            \vfill
            {\huge\bf Metode Numerice - Tema 2}
            \vfill
            Alexandru Sima (312CA) \\
            \small\today
      \end{center}
\end{titlepage}

\fancyhead[L]{Metode Numerice - Tema 2}
\fancyhead[R]{Alexandru Sima(312CA)}
\urlstyle{style=normal}
\tableofcontents
\newpage

\section{Compresia imaginilor folosind SVD (Task 1)}

\subsection{Descriere}
Compresia imaginilor folosind SVD presupune aflarea descompunerii SVD a matricei
asociate imaginii, reținerea primelor $k$ valori singulare și reasamblarea
imaginii.

\subsection{Algoritm}
\begin{itemize}
      \item Se calculează descompunerea SVD a matricei asociate imaginii.
      \item Se rețin primele $k$ valori singulare.
      \item Se asamblează imaginea aproximată, folosind valorile singulare
            reținute.
\end{itemize}

\section{Compresia imaginilor folosind analiza componentelor principale (Task 2)}

\subsection{Descriere}
Compresia imaginilor folosind analiza componentelor principale presupune aflarea
matricei componentelor principale, reținerea primelor $k$ componente principale
și reasamblarea imaginii.

\subsection{Algoritm}
\begin{itemize}
      \item Se normalizează fiecare rând al matricei (adică se scade din fiecare
            pixel media rândului pe care se află, fiecare rând având acum media
            0).
      \item Se construiește matricea Z.
      \item Se calculează matricea de componente principale a matricei Z (V din
            descompunerea SVD).
      \item Se rețin primele $k$ componente principale.
      \item Se reconstruiește matricea aproximativă.
\end{itemize}

\section{Compresia imaginilor folosind analiza componentelor principale și a
  matricei de covarianță (Task 3)}

\subsection{Descriere}
Compresia imaginilor presupune aflarea matricei de covarianță asociată imaginii,
reținerea vectorilor proprii asociați celor mai mari valori proprii și
reasamblarea imaginii.

\subsection{Algoritm}
\begin{itemize}
      \item Se normalizează fiecare rând al matricei.
      \item Se calculează matricea de covarianță a imaginii.
      \item Se calculează vectorii și valorile proprii ale matricei de
            covarianță.
      \item Se sortează vectorii proprii descrescător după valorile proprii.
            Pentru aceasta, se sortează valorile proprii și se păstrează indicii
            originali, apoi se indexează matricea vectorilor proprii folosind
            acești indici.
      \item Se rețin primii $k$ vectori proprii.
      \item Se asamblează imaginea aproximată, folosind vectorii proprii
            reținuți.
      \item Se denormalizează imaginea.
\end{itemize}

\section{Recunoașterea cifrelor scrise de mână (Task 4)}

\subsection{Fișiere}
\begin{itemize}
      \item $prepare_data$: Încarcă setul de date MNIST și îl împarte în
            setul de antrenare și cel de testare.
      \item $visualise_image$: Afișează o imagine din setul de date.
      \item $magic_with_pca$: Calculează matricea $train$, aproximând, folosind
            matricea de covarianță, imaginile de antrenament.
      \item $prepare_photo$: Pregătește imaginea pentru clasificare.
      \item $KNN$: Calculează distanța euclidiană între imaginea de test și
            matricea de antrenament, predicția fiind cifra asociată medianei
            celor mai mari $k$ distanțe.
      \item $classifyImage$: Clasifică imaginile din baza de date MNIST,
            asamblând funcțiile de mai sus.
\end{itemize}

\subsection{Descriere}
\begin{itemize}
      \item Pentru a realiza clasificări ale imaginilor, se antrenează un model
            folosind o parte a setului de date, aproximându-se imaginile
            folosind descompunerea în valori singulare a matricelor de
            covarianță ale matricelor imaginilor.
      \item Pentru clasificările propriu-zise, se calculează distanțele între
            imaginea de clasificat și cele de test, imaginea revenind clasei
            imaginii față de care distanța este mediana tutoror distanțelor.
\end{itemize}

\subsection{Algoritm}
\begin{itemize}
      \item Se încarcă setul de date.
      \item Se împarte în date de antrenament și de test.
      \item Se calculează matricea de antrenament, aproximându-se imaginile.
      \item Se clasifică imaginile de test, calculându-se distanțele până la
            imaginile din matricea de antrenament și se alege clasa medianei
            primelor $k$ distanțe.
\end{itemize}

\end{document}
